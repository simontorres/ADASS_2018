% super simple template for automated 2018 ADASS manuscript generation from the registration entry
% most comments have been removed, see the ADASS_template.tex for a fully commented version

% Version 4-nov-2018 (Peter Teuben)

\documentclass[11pt,twoside]{article}
\usepackage{asp2014}

\aspSuppressVolSlug
\resetcounters

\bibliographystyle{asp2014}


\markboth{Torres, and Author2}{An on-site data reduction pipeline for the Goodman Spectrograph}      % remove/add authors as you need

\begin{document}

\title{An on-site data reduction pipeline for the Goodman Spectrograph}


\author{Simón~Torres,$^1$ and Sample~Author2$^2$
  \affil{$^1$SOAR Telescope, Institution City, State/Province, Country; \email{storres@ctio.noao.edu}}
  \affil{$^2$Institution Name, Institution City, State/Province, Country}}           % remove/add authors as you need


\paperauthor{Simón~Torres}{storres@ctio.noao.edu}{ORCID}{SOAR Telescope}{Author1 Department}{City}{State/Province}{Postal Code}{Country}
% remove/add authors as you need
\paperauthor{Sample~Author2}{Author2Email@email.edu}{ORCID_Or_Blank}{Author2 Institution}{Author2 Department}{City}{State/Province}{Postal Code}{Country}

% leave these commented for the editors to enable them
%\aindex{Torres,~S.}
%\aindex{Coauthor,~A.}          % remove and add as you need
  
\begin{abstract}

The Goodman Spectroscopic Pipeline (GSP) is reaching some maturity and
behaving in a stable manner. Though its development continues, we have
started a parallel effort to develop a real-time version of the GSP,
which aims at obtaining fully reduced data within seconds after the spectrum
has been obtained at the telescope.  Most of the required structure, algorithms and
processes already exist with the offline version of the GSP. The real-time or online
version differs in its requirements for flow control, calibration files, image combination, reprocessing, observing logging assistance, etc.
Here we present results obtained with the offline version of GSP with various instrument setups, and outline the route for implementation of the real time, online version.
  
\end{abstract}

\section{Introduction}

Your abstract currently has 769 characters. For more than 1000 it's possibly too long. Just sayin'
Since this paper was written by some python code, ignore that warning, but better edit most of this rubbish away.


\section{The Template}

To use this 2018 template instead of the ADASS\_template, copy this file to your given paper, e.g. O3-1.tex, P5-2.tex, B4.tex, F3.tex, I.tex, place the
paper type in the Makefile, review the Makefile, and hit ``make'' and hope for the best.  If that runs into trouble, check if your version of
latex uses a different calling sequence.  Some instructions are in the Makefile.

\section{Figures}

This template has no figures.
Look for the larger template and Makefile how to do this.

\section{References}

This template has no bibtex file. 
Look for the larger template and Makefile how to do this. By default the Makefile will create
an empty P4-18.bib. When you add references to this, uncomment  the 
line \verb+\bibliography+  below, make use ``make'' to
create your beautifully looking PDF.

% \bibliography{P4-18}

\end{document}

