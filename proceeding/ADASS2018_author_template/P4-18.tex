% super simple template for automated 2018 ADASS manuscript generation from the registration entry
% most comments have been removed, see the ADASS_template.tex for a fully commented version

% Version 4-nov-2018 (Peter Teuben)

\documentclass[11pt,twoside]{article}
\usepackage{asp2014}

\aspSuppressVolSlug
\resetcounters

\bibliographystyle{asp2014}


\markboth{Torres-Robledo, and Brice\~no}{A Real-Time Data Reduction Pipeline for the Goodman Spectrograph}      % remove/add authors as you need

\begin{document}

\title{A Real-Time Data Reduction Pipeline for the Goodman Spectrograph}


\author{Sim\'on~Torres-Robledo,$^1$ and C\'esar~Brice\~no$^{1,2}$
  \affil{$^1$SOAR Telescope, La Serena 1700000, Chile; \email{storres@ctio.noao.edu}}
  \affil{$^2$Cerro Tololo Interamerican Observatory, Casilla 603, La Serena 1700000, Chile}}           % remove/add authors as you need


\paperauthor{Sim\'on~Torres-Robledo}{storres@ctio.noao.edu}{ORCID}{SOAR Telescope}{Author1 Department}{La Serena}{Regi\'on de Coquimbo}{Postal Code}{Chile}
% remove/add authors as you need
\paperauthor{C\'esar~Brice\~no}{cbriceno@ctio.noao.edu}{ORCID_Or_Blank}{SOAR Telescope}{Author2 Department}{La Serena}{Regi\'on de Coquimbo}{Postal Code}{Chile}

% leave these commented for the editors to enable them
%\aindex{Torres,~S.}
%\aindex{Brice\~no,~C.}          % remove and add as you need
  
\begin{abstract}

The Goodman Spectroscopic Pipeline is reaching some maturity and
behaving in a stable manner. Though its improvement continues, we have
started a parallel effort to develop a real-time version,
with the goal of obtaining fully reduced spectra seconds after the data
has been obtained at the telescope.  Most of the required structure, algorithms and
processes already exist with the offline version. The real-time version differs in its requirements for flow control, calibration files, image combination, reprocessing, observing logging assistance, etc.
Here we present an outline of the route for implementation of a real time online version of the Goodman spectroscopic pipeline.
  
\end{abstract}

\section{Introduction}

The 4-m Southern Astrophysical Research telescope \emph{SOAR} telescope, located on Cerro Pach\'on, northern Chile, currently has as it most used instrument the \emph{Goodman High Throughput Spectrograph} (GHTS), an imaging spectrograph developed by the University of North Carolina \citet{2004SPIE.5492..331C}. SOAR operations are run in classical mode, with approved proposals scheduled on specific dates. Though observers can go up to the summit, most often they observe from a remote location via an Internet connection, through a VPN tunnel, accessing the instrument software with a VNC client. With the advent of the new generation of survey telescopes such as LSST, SOAR is aiming at becoming a prime follow up facility for transients and Time Domain events. To this effect, a project has been set up in collaboration with the National Optical Astronomical Observatory (NOAO) and Las Cumbres Observatory (LCO), to automate various processes in order to make observations more efficient, allow the telescope to respond faster to incoming alerts, and interface smoothly with the existing robotic scheduling technology developed by LCO. 

The Goodman Spectroscopic Pipeline (GSP) is part of this effor, with an online version envisioned, working automatically and capable of delivering science-ready spectra seconds after the shutter has closed. This is particularly important when scientists need to make real time decisions like whether to obtain additional spectra of a given  object that may be on the rise, or fading. However, at the moment of writing GSP exists only as an offline version, a one-line command that can be run once the observing night has finished. The user connects via VPN and VNC to a dedicated data reduction machine on which we have the latest version of the GSP. 
%Though VNC connections are relatively reliable and simple to setup they come with certain flaws, such as large bandwidth requirement, zero user management capabilities and with it, security flaws, non-responsive layout and usually requires the use of large screens. 
In this contribution we describe the requirements for an online implementation of the GSP using web technology with secure authentication and responsive layout, using modern, well proven technology.

\section{Goodman Spectroscopic Pipeline: offline version status and performance}

The GSP is based on a two-step process. \verb=redccd= does the basic CCD data reduction, common for imaging and spectroscopy, with some small differences for spectroscopy. \verb=redspec= does spectroscopic-specific data reduction \citet{2013pss2.book...35M}, including automatic wavelength calibration. Processing a typical full night takes between three and five minutes.

%\subsection{Wavelength Calibration}
From the start of this project, we aimed for an automated, unsupervised, wavelength calibration of the spectra. After experimenting obtaining wavelength calibrations with several methods, including trying out an interactive module, we found that the best solution was to use a catalog of templates, a collection of comparison lamp spectra taken with our own hollow cathode lamps. This approach provides good, reliable solutions without the need for any manual interaction from the user. The GHTS is a highly configurable instrument, which means that the camera and grating angle can be adjusted to almost any combination of values, yielding a wide range of possible wavelength coverage options. This flexibility leads to practical problems when trying to setup a library of comparison lamps for various modes, because of the large number of possible wavelength ranges.  Therefore, we decided to setup the standard spectral lamp library limited to the most often used modes, for the more frequently requested gratings. It is still possible to use the instrument in \emph{Custom} mode (in which the user defines the central wavelength for the Littrow mode) but such custom modes do not have a corresponding template in the library, for obvious reasons. 
%Still, if the user generates the templates, then the GSP will work with that specific setup. 
There are seven gratings available at present for the Goodman spectrograph: 400, 600, 930, 1200, 1800, 2100, and 2400 $l/mm$; for the last three there is no fixed mode defined, but rather they are normally used in Littrow mode. We built standard lamp spectra for the two most used modes of the 400 line grating, and for several modes of the 600, 930 and 1200 line gratings.

\begin{table}[!ht]
\caption{Sample of spectroscopic modes definition for the 930 $l/mm$ grating}
\begin{tabular}{|c|c|c|c|c|}
\hline
    Grating  & Dispersion & Coverage& Max R @ 5500nm & Blocking Filter \\
    (lines/mm) & (\AA/pixel)  &  (\AA)  & (3 pix with 0.46'' slit) & \\
\hline
    930 & 0.42 & M1: 300-470& 4450 & -- \\
    & & M2: 385-555 & & -- \\
    & & M3: 470-640 & & GG-385\\
    & & M4: 555-725 & & GG-495\\
    & & M5: 640-810 & & GG-495\\
    & & M6: 725-895 & & OG-570\\
\hline
\end{tabular}
\end{table}

The lamps in the library are not linearized because the raw spectra coming out of the GHTS are non-linear in wavelength space, therefore reference wavelength solutions need to be non-linear. All the emission lines detected in the lamp spectrum are recorded in the header, together with their corresponding wavelength value as obtained from the fit of the mathematical model used to describe the solution. 
%Another technique we attempted is to use the grating equation to obtain a theoretical first approximation to the wavelength solution. Then, using line list information such as that provided by NIST (SIMON, FAVOR EXPLICAR EL ACRONIMO "NIST"), build a template to cross correlate with the actual comparison lamp. The problem with this second approach is that line list does not have accurate line intensity information; a usual situation when the solution failed is that some selected lines did not exist in the observed comparison lamp \citep[see][for a list of other aspects that affect the resulting lamp spectrum]{2018A&A...618A.118S}.

Performance-wise the results have been quite satisfactory. 
%though the precision depends on many variables such as lamp quality, combination of grating and spectroscopic mode (this will affect the number of lines present in the lamp spectrum), and ultimately the slit size. 
Overall, we obtain root-mean square (RMS) values similar to those obtained  using IRAF. For instance using the 930 l/mm grating in the M2 mode the RMS error of the wavelength solution was 0.281\AA.
%\subsection{Room for Improvement}
 %Nevertheless, there is still room for improvement, for instance, addition of a flux calibration module, and replacing the C-language cosmic ray rejection routine DCR (\citet{2004PASP..116..148P}) with a Python routine. Despite DCR being a great tool for removing cosmic rays, being written in C required wrappers, and more importantly, complicates installation of GSP, particularly with PIP. Also our goal was to use as much Astropy code as possible and this was one exception due to how well it behaved. Astroscrappy is an implementation of LACosmic (\citet{2001PASP..113.1420V}) which is wrapped by \verb=ccdproc.cosmicray_lacosmic=. At the moment we have both DCR and Astroscrappy running. It is a matter of fine tuning the parameters in Astroscrappy in order to obtain the same results as with DCR, though its takes some effort to arrive at the optimum values. 

\section{Goodman Spectroscopic Pipeline Real-Time version: Design Constraints}

For the live data reduction pipeline we want to do away with the VNC system and move to a web-based service, that would allow secure authentication. Also, the offline version of the GSP exists as a single Python package; for the live version  we will need several other components.

\articlefigure{P4-18_f1.eps}{livepipelineapi}{Simplified schematic representing the hierachical structure and relation between components for the proposed design.
Several components were intentionally omited due to spacing.}

\subsection{Data Reduction Package}

This is based on the offline GSP Python package. The modifications required for the real-time version are intended to enable asynchronous data processing as well as control, this could be achieved by adding a private REST API that would enable modifying the pipeline's settings needed to operate under different observing strategies. It has to be relatively simple and able to work independently and automatically, allowing intervention by request from the Public API. 
The allowed control hooks in the private API should be minimal so that authentication is not required, though the access should be constrained to the Observatory's private network. Some of the controls that need to be included are: turn on or off file system event watching routines, alter settings, report errors. It should also handle a light database for redundancy in case the web server fails.
In terms of data processing most of the routines are already implemented in the offline version, but there are a couple of things still missing, such as optimal extraction \citet{1989PASP..101.1032M} and  \citet{1986PASP...98..609H} flux calibration, deblending of multiple sources, low signal-to-noise extraction.

As is explained in \citet{P9-115_adassxxvii} our philosophy is to rely as much as possible on Astropy's code  \citet{2013A&A...558A..33A} and \citet{2018AJ....156..123A} therefore, some of the code we had to develop ourselves because there was nothing equivalent implemented yet in Astropy. We plan to add these pieces of code as our contribution to the appropriate Astropy Package.

\subsection{Public REST API}
Publishing a web application is not a simple task, with security been the biggest point of concern, but also that the application be stable and reliable. Fortunately, there are plenty of tools that allow us to simplify these tasks, in fact, most of them, which is important because we don't have the resources to hire an entire team of developers experts on these very specific tools. The scope of the services provided by the public API should be end-user oriented only, such as, secure authentication and channeling communications with the private API.

\subsection{Web Front End}
A highly responsive website is favored over a local GUI for one simple reason: most of the GHTS users are working remotely, while local users will still benefit from it. Though more difficult to implement, there are several benefits that come with a web font end, for instance: adaptive layout, user experience less dependant on connection quality, less bandwidth usage and of course taking advantage of the interactive experience that web technology allows.

We have not yet decided what are the tools (stack) that we will use, but the criteria for selecting them are: being well documented and easily testable. By \emph{easily} we mean that there have to be good tools available and a good testing philosophy behind its development. Testability is highlighted here but it applies to all the code of our project.


\acknowledgements The authors would like to acknowledge the important contribution from Bruno Quint, David Sanmartim and Tina Armond. This research made use of Astropy, a community-developed core Python package for Astronomy \citet{2013A&A...558A..33A} and \citet{2018AJ....156..123A} This work has been developed at the Southern Astrophysical Research (SOAR) telescope, which is a joint project of the Minist\'erio da Ci\^encia, Tecnologia, Inova\c{c}\~ao e Comunica\c{c}\~oes (MCTIC) do Brasil, the U.S. National Optical Astronomy Observatory (NOAO), the University of North Carolina at Chapel Hill (UNC), and Michigan State University (MSU). 

\bibliography{P4-18}

\end{document}

